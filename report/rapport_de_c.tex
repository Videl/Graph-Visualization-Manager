\documentclass[12pt,a4paper]{report}
\usepackage[utf8]{inputenc}
\usepackage[T1]{fontenc}
\usepackage{amsmath}
\usepackage{amsfonts}
\usepackage{amssymb}
\usepackage[urlbordercolor={0 0 1}]{hyperref}
\usepackage[left=2cm,right=2cm,top=2cm,bottom=2cm]{geometry}
\usepackage{graphicx} 
\usepackage{array}
 
\usepackage[francais]{babel}

\makeatletter
\def\maketitle{
  \null
  \begin{flushleft}
   \includegraphics[width=6cm]{telecom_nancy.png}
  \end{flushleft}
  \vfill
  \begin{center}\leavevmode
    \normalfont
    {\LARGE \@title\par}
    {\Large \@date\par} 
    \vskip 1cm
    
  \end{center}
  \vfill
  \hfill
  \begin{flushright}
    {\Large \@author\par}
  \end{flushright}
  \cleardoublepage
  }
\makeatother
\date{2012-2013}
\author{Manon Signoret G32\\Thibaut Smith G32}
\title{Projet CSH\\Visualisation d'un r\'{e}seau de transport}

\renewcommand*\thesection{\arabic{section}} % jolies Sections

\begin{document}

\thispagestyle{empty}
\maketitle
\pagebreak

\section{Introduction}
Nous avons choisi le sujet 4, un programme permettant de visualiser un r\'{e}seau de transport. Le programme permet de d\'{e}finir et visualiser un r\'{e}seau de transport d'un type de marchandise : on peut d\'{e}finir l'espace g\'{e}ographique de travail de mani\`{e}re al\'{e}atoire (pour les tests), ou manuellement. De plus, on peut aussi g\'{e}n\'{e}rer des chemins entre plusieurs d\'{e}pôts dans un ordre choisi. Le programme est \'{e}crit en C, et fonctionne enti\`{e}rement en ligne de commande\footnote{\`{a} v\'{e}rifier \`{a} cause de Graphviz}. Le code est disponible sur \href{https://github.com/Videl/Graph-Visualization-Manager}{GitHub}.


\section{Cahier des charges}
Le programme avait plusieurs objectifs \`{a} r\'{e}aliser. En voici la liste, avec les solutions choisies pour atteindre l'objectif.

\begin{center}
  \begin{tabular}{|m{7cm}|m{7cm}|}
    \hline
    \textbf{Objectifs} & \textbf{Solutions} \\
    \hline
    Repr\'{e}sentation des graphes dans la m\'{e}moire       & Structures de \textbf{Graphviz} \\
    \hline
    Saisie d'informations venant de l'utilisateur    & Entr\'{e}es s\'{e}curis\'{e}es (\textbf{gets}) \\
    \hline
    Repr\'{e}sentation des d\'{e}pôts                        & Structure et listes chaîn\'{e}es simple\footnote{\`{a} v\'{e}rifier \`{a} cause de Graphviz} \\
    \hline
    Algorithme de calcul du chemin optimum           & \href{https://en.wikipedia.org/wiki/Travelling_salesman_problem}{Travelling salesman problem} \\
    \hline
    Possibilit\'{e} de d\'{e}finir le r\'{e}seau (nombre de d\'{e}p\^ots, leur localisation, les liaisons existantes)			     &	IHM \\
    \hline
    Choix du d\'{e}p\^ot source/destination(s)			 & IHM \\
    \hline
    Co\^ut de chaque liaison						 & IHM \\
    \hline
  	Visualisation du r\'{e}seau							 & Graphviz ou fonctions maison ?? \\
    \hline
    Interface Homme-Machine (IHM)                    & Terminal (max 80 colonnes) \\
    \hline
    
  \end{tabular}
\end{center}

\section{Choix de conceptions}
Ne pas oublier de mettre quelques screens

\section{D\'{e}roulement du projet}
Nous avons choisi d'utiliser Git pour ce projet, \'{e}tant donn\'{e} que nous avions d\'{e}j\`{a} utilis\'{e} SVN\footnote{Il ne manque plus que Bazaar, Mercurial, etc!}. La forge de l'\'{e}cole ne supportant pas Git, nous avons choisi d'h\'{e}berger notre code source sur \href{https://github.com/Videl/Graph-Visualization-Manager}{GitHub}.
\paragraph{Problèmes survenus}
Tout d'abord, nous avons eu du mal \`{a} comprendre la documentation, et nous n'\'{e}tions pas s\^ur encore de quoi faire, et comment. Ce qui a retard\'{e} le projet. Nous sommes donc all\'{e} chercher de l'aide d'abord vers un binôme en particuler\footnote{Clemence Henry, Pierre Monnin}, et ensuite sur Internet. Cela nous a d\'{e}bloqu\'{e}, et \`{a} partir de ce moment, le projet a pu commencer.

\section{Nombre d'heures}
Nous avons pass\'{e} un nombre important d'heures sur la compr\'{e}hension de la documentation de Graphviz avant de savoir quoi faire, Graphviz fonctionnant avec ses propres structures.

\begin{tabular}{|c|c|}
  \hline
  Travail & Heures \\
  \hline
  Doc' Graphviz & ~5H \\
  \hline
  D\'{e}veloppement	&	~10H \\
  \hline
  Tests	&	~5H \\ 
  \hline
  Rapport	&	~3H \\
  \hline
\end{tabular}

\section{Sources}

\end{document}
